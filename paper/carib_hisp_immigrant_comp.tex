% Options for packages loaded elsewhere
\PassOptionsToPackage{unicode}{hyperref}
\PassOptionsToPackage{hyphens}{url}
\PassOptionsToPackage{dvipsnames,svgnames,x11names}{xcolor}
%
\documentclass[
]{article}

\usepackage{amsmath,amssymb}
\usepackage{setspace}
\usepackage{iftex}
\ifPDFTeX
  \usepackage[T1]{fontenc}
  \usepackage[utf8]{inputenc}
  \usepackage{textcomp} % provide euro and other symbols
\else % if luatex or xetex
  \usepackage{unicode-math}
  \defaultfontfeatures{Scale=MatchLowercase}
  \defaultfontfeatures[\rmfamily]{Ligatures=TeX,Scale=1}
\fi
\usepackage{lmodern}
\ifPDFTeX\else  
    % xetex/luatex font selection
\fi
% Use upquote if available, for straight quotes in verbatim environments
\IfFileExists{upquote.sty}{\usepackage{upquote}}{}
\IfFileExists{microtype.sty}{% use microtype if available
  \usepackage[]{microtype}
  \UseMicrotypeSet[protrusion]{basicmath} % disable protrusion for tt fonts
}{}
\makeatletter
\@ifundefined{KOMAClassName}{% if non-KOMA class
  \IfFileExists{parskip.sty}{%
    \usepackage{parskip}
  }{% else
    \setlength{\parindent}{0pt}
    \setlength{\parskip}{6pt plus 2pt minus 1pt}}
}{% if KOMA class
  \KOMAoptions{parskip=half}}
\makeatother
\usepackage{xcolor}
\setlength{\emergencystretch}{3em} % prevent overfull lines
\setcounter{secnumdepth}{-\maxdimen} % remove section numbering
% Make \paragraph and \subparagraph free-standing
\makeatletter
\ifx\paragraph\undefined\else
  \let\oldparagraph\paragraph
  \renewcommand{\paragraph}{
    \@ifstar
      \xxxParagraphStar
      \xxxParagraphNoStar
  }
  \newcommand{\xxxParagraphStar}[1]{\oldparagraph*{#1}\mbox{}}
  \newcommand{\xxxParagraphNoStar}[1]{\oldparagraph{#1}\mbox{}}
\fi
\ifx\subparagraph\undefined\else
  \let\oldsubparagraph\subparagraph
  \renewcommand{\subparagraph}{
    \@ifstar
      \xxxSubParagraphStar
      \xxxSubParagraphNoStar
  }
  \newcommand{\xxxSubParagraphStar}[1]{\oldsubparagraph*{#1}\mbox{}}
  \newcommand{\xxxSubParagraphNoStar}[1]{\oldsubparagraph{#1}\mbox{}}
\fi
\makeatother


\providecommand{\tightlist}{%
  \setlength{\itemsep}{0pt}\setlength{\parskip}{0pt}}\usepackage{longtable,booktabs,array}
\usepackage{calc} % for calculating minipage widths
% Correct order of tables after \paragraph or \subparagraph
\usepackage{etoolbox}
\makeatletter
\patchcmd\longtable{\par}{\if@noskipsec\mbox{}\fi\par}{}{}
\makeatother
% Allow footnotes in longtable head/foot
\IfFileExists{footnotehyper.sty}{\usepackage{footnotehyper}}{\usepackage{footnote}}
\makesavenoteenv{longtable}
\usepackage{graphicx}
\makeatletter
\def\maxwidth{\ifdim\Gin@nat@width>\linewidth\linewidth\else\Gin@nat@width\fi}
\def\maxheight{\ifdim\Gin@nat@height>\textheight\textheight\else\Gin@nat@height\fi}
\makeatother
% Scale images if necessary, so that they will not overflow the page
% margins by default, and it is still possible to overwrite the defaults
% using explicit options in \includegraphics[width, height, ...]{}
\setkeys{Gin}{width=\maxwidth,height=\maxheight,keepaspectratio}
% Set default figure placement to htbp
\makeatletter
\def\fps@figure{htbp}
\makeatother
% definitions for citeproc citations
\NewDocumentCommand\citeproctext{}{}
\NewDocumentCommand\citeproc{mm}{%
  \begingroup\def\citeproctext{#2}\cite{#1}\endgroup}
\makeatletter
 % allow citations to break across lines
 \let\@cite@ofmt\@firstofone
 % avoid brackets around text for \cite:
 \def\@biblabel#1{}
 \def\@cite#1#2{{#1\if@tempswa , #2\fi}}
\makeatother
\newlength{\cslhangindent}
\setlength{\cslhangindent}{1.5em}
\newlength{\csllabelwidth}
\setlength{\csllabelwidth}{3em}
\newenvironment{CSLReferences}[2] % #1 hanging-indent, #2 entry-spacing
 {\begin{list}{}{%
  \setlength{\itemindent}{0pt}
  \setlength{\leftmargin}{0pt}
  \setlength{\parsep}{0pt}
  % turn on hanging indent if param 1 is 1
  \ifodd #1
   \setlength{\leftmargin}{\cslhangindent}
   \setlength{\itemindent}{-1\cslhangindent}
  \fi
  % set entry spacing
  \setlength{\itemsep}{#2\baselineskip}}}
 {\end{list}}
\usepackage{calc}
\newcommand{\CSLBlock}[1]{\hfill\break\parbox[t]{\linewidth}{\strut\ignorespaces#1\strut}}
\newcommand{\CSLLeftMargin}[1]{\parbox[t]{\csllabelwidth}{\strut#1\strut}}
\newcommand{\CSLRightInline}[1]{\parbox[t]{\linewidth - \csllabelwidth}{\strut#1\strut}}
\newcommand{\CSLIndent}[1]{\hspace{\cslhangindent}#1}

\usepackage{pdflscape}
\usepackage{caption}
\usepackage{booktabs}
\usepackage{threeparttable}
\usepackage{fancyhdr}
\usepackage{lastpage}
\usepackage{geometry}
\usepackage{array}
\usepackage{calc}
\makeatletter
\@ifpackageloaded{caption}{}{\usepackage{caption}}
\AtBeginDocument{%
\ifdefined\contentsname
  \renewcommand*\contentsname{Table of contents}
\else
  \newcommand\contentsname{Table of contents}
\fi
\ifdefined\listfigurename
  \renewcommand*\listfigurename{List of Figures}
\else
  \newcommand\listfigurename{List of Figures}
\fi
\ifdefined\listtablename
  \renewcommand*\listtablename{List of Tables}
\else
  \newcommand\listtablename{List of Tables}
\fi
\ifdefined\figurename
  \renewcommand*\figurename{Figure}
\else
  \newcommand\figurename{Figure}
\fi
\ifdefined\tablename
  \renewcommand*\tablename{Table}
\else
  \newcommand\tablename{Table}
\fi
}
\@ifpackageloaded{float}{}{\usepackage{float}}
\floatstyle{ruled}
\@ifundefined{c@chapter}{\newfloat{codelisting}{h}{lop}}{\newfloat{codelisting}{h}{lop}[chapter]}
\floatname{codelisting}{Listing}
\newcommand*\listoflistings{\listof{codelisting}{List of Listings}}
\makeatother
\makeatletter
\makeatother
\makeatletter
\@ifpackageloaded{caption}{}{\usepackage{caption}}
\@ifpackageloaded{subcaption}{}{\usepackage{subcaption}}
\makeatother

\ifLuaTeX
  \usepackage{selnolig}  % disable illegal ligatures
\fi
\usepackage{bookmark}

\IfFileExists{xurl.sty}{\usepackage{xurl}}{} % add URL line breaks if available
\urlstyle{same} % disable monospaced font for URLs
\hypersetup{
  pdftitle={Sociodemographic Comparison of Caribbean Hispanic Older Adult Immigrants in the U.S. and Origin Countries},
  pdfauthor={William H. Dow; Chris Soria; Henry T. Dow},
  colorlinks=true,
  linkcolor={blue},
  filecolor={Maroon},
  citecolor={Blue},
  urlcolor={Blue},
  pdfcreator={LaTeX via pandoc}}


\title{Sociodemographic Comparison of Caribbean Hispanic Older Adult
Immigrants in the U.S. and Origin Countries}
\author{William H. Dow \and Chris Soria \and Henry T. Dow}
\date{2025-01-15}

\begin{document}
\maketitle
\begin{abstract}
Caribbean and adjacent Latin American countries are key sources of
Hispanic immigrants to the U.S. There has been rapid growth in the older
adult Hispanic populations both among immigrants in the U.S. and in
their home countries of emigration. This paper supports hypothesis
generation for international comparative Hispanic aging studies by
comparing older adult sociodemographic characteristics of U.S.
immigrants versus those in sending countries. The analysis also provides
context for the global family of health and retirement studies in the
region including the ongoing Caribbean American Dementia and Aging Study
(CADAS) which is collecting harmonized data on healthy aging in Puerto
Rico, Dominican Republic, and Cuba. We analyze census microdata from
these countries along with other major Hispanic Caribbean-adjacent
sending countries including Mexico, Colombia, El Salvador, Guatemala,
and Honduras. We compare older adults in these sending countries to
country-specific immigrant samples in the U.S. American Community
Survey, focusing on socioeconomic differences such as education, as well
as marital status and co-residence patterns related to caregiver
availability. We also examine differences by citizenship and immigration
age to further explore immigrant selectivity patterns. The highly varied
experiences of these cohorts will help inform future comparative
research on Hispanic healthy aging.
\end{abstract}


\setstretch{1.2}
\subsection{Introduction and Background TODO: Rewrite this whole
intro}\label{sec-intro}

Latin American and Caribbean Countries (LACCs) are key sources of
Hispanic immigrants to the United Sates (Passel 2024). In 2022, people
of Mexican origin made up nearly 60\% of the U.S. Hispanic population,
totaling about 37.4 million. Puerto Ricans were the next largest group
at 5.9 million, with an additional 3.2 million living on the island.
Salvadorans, Cubans, Dominicans, Guatemalans, Colombians, and Hondurans
each have populations exceeding 1 million in the United States
(Noe-Bustamante 2023).

These immigrant populations include a rapidly growing subgroup who are
aged 65 and above, among whom there is wide variation in socioeconomic
and caregiving resources. In this paper we explore sociodemographic
variation of U.S. older adult immigrants by country and cohort of
emigration, and compare these U.S. immigrants to the corresponding
cohorts of older adults in their home countries of emigration.

The paper is designed to support hypothesis generation for international
comparative Hispanic aging studies. This includes providing background
context for the global family of health and retirement studies in the
region such as the ongoing Caribbean American Dementia and Aging Study
(CADAS) which is collecting harmonized data on healthy aging in Puerto
Rico, Dominican Republic, and Cuba (Llibre-Guerra et al. 2021). We
analyze census microdata from these countries along with other major
Hispanic Caribbean-adjacent sending countries including Mexico,
Colombia, El Salvador, Guatemala, and Honduras. We compare older adults
in these sending countries to country-specific immigrant samples in the
U.S. American Community Survey, focusing on socioeconomic differences
such as education, as well as marital status and co-residence patterns
related to caregiver availability. We also examine differences by
citizenship and immigration age to further explore immigrant selectivity
patterns. The highly varied experiences of these cohorts will help
inform future comparative research on Hispanic healthy aging.

\subsection{Literature Review}\label{sec-lit}

International migration patterns are shaped by a complex interplay of
factors, including labor market demands, educational opportunities, and
political and economic instability (McAuliffe, Bauloz, and Kitimbo 2024;
Valentine et al. 2017). In the North American context, Massey et
al.~posit that established social networks play a pivotal role in
migration dynamics, with family reunification serving as a primary
motivator (Silva and Massey 2014). This reunification imperative creates
a self-perpetuating cycle of migration flows (Massey et al. 1994). Age
also significantly influences migration patterns, with younger
individuals typically dominating migrant populations. Mexican migrants
exemplify this trend; despite recent cohorts showing a slight increase
in average age, they remain predominantly youthful(Angel, Vega, and
López-Ortega 2017). This persistent age pattern underscores migration's
enduring appeal to younger generations, even as sending countries'
overall populations age {[}TODO: Will, do you know any good papers to
cite here? Maybe an HCAP paper?{]}.

Other LACCs similarly exhibit diverse migration drivers and patterns,
with specific political agreements further shaping these movements and
creating unique migration landscapes across the region. Puerto Rico's
special status as a U.S. territory since 1898, for instance, has
facilitated legal entry to the U.S. mainland for its
residents\footnote{Despite their high out-migration rate, Puerto Ricans
  in the United States send less money than Dominicans and Mexicans to
  their relatives back home. Possible explanations for this are the
  extensive public support system in place on the island and relatively
  higher standard of living compared to other LACCs. (Duany 2010)}.
Likewise, the Cuban Adjustment Act of 1966 and subsequent policies, such
as the ``wet foot, dry foot'' policy (1995-2017), have significantly
influenced Cuban migration to the United States (Duany 2017). These
examples illustrate how political arrangements can create facilitated
migration channels, potentially leading to what migration theorists term
``migration systems'' or ``transnational social spaces'' (Kritz, Lim,
and Zlotnik 1992).

Economic factors underpin much of the migration from Latin America to
the United States, exemplifying classic push-pull migration theory
(Hanson, Orrenius, and Zavodny 2023). Economic insecurity in Latin
American countries acts as a push factor (Capielo Rosario et al. 2023;
Larotta Silva 2019), while periods of US economic growth further amplify
wage differentials between the two regions (Bahar 2024). For instance,
Colombian and Guatemalan workers in the U.S. earn \$288-\$299 for every
\$100 earned by their counterparts at home, while Nicaraguan migrants
see an even larger differential (Clemens, Montenegro, and Pritchett
2009)\footnote{Beyond wages, housing availability (Henao, Lis-Gutiérrez,
  and Balaguera 2023) and access to technology (Nevado-Peña, López-Ruiz,
  and Alfaro-Navarro 2019) significantly influence life satisfaction and
  migration decisions (Causa and Pichelmann 2020; Winkler 2016)}.
However, these wage differentials are partially explained by positive
selection, as migrants often possess characteristics associated with
higher productivity and adaptability. This selection bias is further
evidenced by research showing that immigrants tend to have longer life
expectancies than native-born populations (Aldridge et al. 2018).

Alongside economic factors, political instability has been a significant
driver of forced migration from Latin American countries to the United
States, aligning with theories of refugee and asylum migration
(FitzGerald and Arar 2018). Historical events underscore this trend. The
Cuban Revolution of 1959, for instance, generated the largest refugee
movement to the United States in history, resulting in approximately 1.4
million Cuban refugees (Duany 2017)\footnote{This record was recently
  surpassed 2021-23 (González 2024)}. Similarly, the Salvadoran Civil
War (1979-1992) led to the displacement of about 1 million people, many
of whom subsequently migrated to the U.S., often through irregular
channels (Cervantes 2018). Guatemala and Honduras experienced similar
trajectories of political instability, exacerbated by U.S. interventions
in the mid-20th century (Jonas 2018; Pine 2008). These events led to
substantial emigration waves, with Guatemalan immigrants in the U.S.
increasing from 71,000 in 1980 to over 480,000 by 2000, and Honduran
immigrants from 39,000 to 283,000 (Batalova 2021). Recent data
underscores this trend: in the first 11 months of 2023, over 50\% of
approximately 412,000 asylum applications to the Department of Homeland
Security came from Venezuela, Cuba, Colombia, and Nicaragua
({``Nationwide {Encounters} {\textbar} {U}.{S}. {Customs} and {Border}
{Protection}''} 2024).

Much of the literature on Latin American migrants predominantly focuses
on individuals who emigrated during periods of economic distress and
political upheaval. However, there is a notable paucity of research
examining those who opt to remain in their home countries under similar
circumstances. Additionally, there is a shortage of comparative studies
that analyze Latin American migrants in relation to both their
counterparts who stayed in their home countries and themselves prior to
migration, limiting our understanding of the full spectrum of migration
outcomes and selectivity. We addresses this gap by conducting a
comparative analysis of older adults from Latin American countries who
have migrated to the United States and their counterparts who have
remained in their countries of origin. By examining the sociodemographic
characteristics of these populations, this research aims to elucidate
the selectivity of migration processes and their long-term implications.

\subsection{Data and Methods}\label{sec-methods}

Census data for this study were obtained from IPUMS International
(Ruggles et al. 2024). The current analysis draws on the latest
harmonized census data available in IPUMS for Colombia (2005), Cuba
(2012), the Dominican Republic (2010), El Salvador (2007), Guatemala
(2002), Honduras (2001), Mexico (2020), Puerto Rico (2010), and the
United States (2020)\footnote{PR 2010 is a 1\% sample, PR 2020 is 5\%,
  US 2010 is 1\%, and US 2020 is 5\%. The 2010 and 2020 census data
  available through IPUMS International are limited to 1\% and 5\%
  samples, respectively. While the U.S. IPUMS database offers larger
  samples, it lacks certain variables, such as country of birth for the
  Dominican Republic.}. We on individuals aged 60 to 89, as some
datasets, such as Puerto Rico's, top-coded ages at 89 (TODO: update the
justification here). To ensure comparability, we standardized means in
the international censuses based on the U.S. sex-specific age
distribution. We also applied weights provided by IPUMS to make the
samples nationally representative.

Table 1 presents means of sociodemographics of all 65-89 year-olds from
each country's census. The variables included are age, percent
married/cohabiting, and highest educational level obtained (using
internationally standardized categorizations of less than primary,
primary, secondary, and university). Table 2 displays means for each
country's counterparts who have emigrated to the U.S. In addition to the
variables in Table 1, it includes percent English speakers, percent
naturalized citizens, mean age at immigration, and mean years in the
U.S. based on the American Community Survey data. Table 3 shows the same
variables as in Table 2, drawn from the American Community, but
categorized by race/ethnicity and nativity status.

\subsection{Results and Comparative Analysis}\label{sec-results}

TODO: Identify which migrant groups moved to the US at the youngest ages
comparatively Variations in migration motivations and experiences
Disparities in socioeconomic outcomes and integration patterns Unique
challenges faced by specific national groups

Questions: which countries send the most skilled workers, which have the
most educated people? Differences in age, gender, and education levels
among various national groups Geographic distribution and settlement
patterns within the United States Healthcare access and outcomes Housing
conditions Who are the people choose to stay, and why do they stay?

\subsubsection{Hispanic older adults in their native
countries}\label{hispanic-older-adults-in-their-native-countries}

Table 1 shows sex-specific sociodemographic characteristics among older
adults aged 65 to 89, comparing across current country of residence in
the Hispanic Caribbean and adjacent countries. Rates of current
marriage/cohabitation are substantially higher among men than women;
among women, they vary significantly across countries, ranging from 38\%
in the Dominican Republic to 48\% in the U.S. Regarding education, there
are even larger differences between neighboring countries. E.g., among
women in this age group, 76\% of Dominicans have less than primary
education compared with 31\% of Cubans and Puerto Ricans; the highest
rate is Honduras at 86\% and lowest is the U.S. at 4\%. There are
similarly large differences among men . \#\#\# Hispanic older adults as
migrants in the U.S.

Table 2 presents sociodemographics of migrants from these countries
living in the U.S. The youngest groups are men from El Salvador (71.23),
Guatemala (70.78), and Honduras (71.5), who are about two years younger
than their counterparts in their native. The oldest group consists of
Cuban women (73.35), who are slightly younger than U.S. women overall
(73.75), while Cuban men (74.6) are older than the U.S. average for men
(73.13). These differences reflect a combination of differential
mortality rates and varying immigration patterns by birth cohort.
Marital/cohabitation status is also affected by these factors as well as
potentially by different cultural influences on partnership; for many
countries rates are somewhat lower among immigrants in the U.S. For
example, the age-adjusted married/cohabiting rate among women in Cuba is
43\%, compared to 31\% for Cuban-born women immigrants in the U.S.

Many of these migrant groups have low rates of reporting that they speak
English, with women generally less likely to speak English than men. The
lowest English speaking rates are among migrants from the Dominican
Republic, where only 60\% of female and 70\% of male migrants speak
English, and Mexico, with 65\% of female and 74\% of males speaking
English. In contrast, migrants from Cuba (71\% of females and 78\% of
males), Colombia (83\% of females and 88\% of males), and Guatemala
(79\% of females and 87\% of males) have higher English speaking rates.

In terms of citizenship, females are slightly more likely to become
naturalized U.S. citizens than men. Typically, men emigrate to the U.S.
at a younger age, about two years earlier on average. For example,
females from Honduras arrive at an average age of 36.48, while males
arrive at 34.42, often resulting in longer U.S. residency for
men---Honduran men report an average of 37.16 years compared to women's
36.17 years. However, Guatemala is an exception; despite migrating at a
younger age (32.05), Guatemalan men have spent fewer years in the U.S.
(38.75) compared to women (39.26).

Men are more likely to have a college degree across all groups:
Guatemalan men (12\%) compared to women (8\%), Honduran men (14\%)
versus women (11\%), and Salvadoran men (10\%) against women (5\%). The
largest gender gap is among Colombian migrants, with 24\% of men holding
a college degree compared to 16\% of women. The least educated groups
are Mexican migrants (6\% of men and 4\% of women with college degrees)
and those from the Dominican Republic (9\% for men and 7\% for women).
These relative gender patterns are similar to their counterparts living
in their native countries (Table 1), however overall the U.S. migrants
have markedly higher education than those still living in their
countries of birth.

\subsubsection{Race/ethnicity and nativity in
U.S.}\label{raceethnicity-and-nativity-in-u.s.}

Table 3 is similar to table 2, but now looks by racial category (Black,
White, and Other racial identities), Hispanic, and native-born category.
Average ages across these groups are quite similar. Both Hispanic
migrant and native Blacks are much less likely to be married or
cohabitating, with rates of 28\% and 27\%, respectively. In contrast,
individuals identifying as White, whether Hispanic migrant or native,
are more likely to be married or cohabitating. Black migrants are
slightly more likely to speak English but are less likely to become
naturalized citizens. They also tend to immigrate at an older age and
spend fewer years in the U.S., while Whites and those of other Hispanic
races have similar immigration patterns. White Hispanic migrants have
slightly longer U.S. residency. Interestingly, the college degree gap
between Black and White migrants is nonexistent; both groups are equally
likely to earn a degree. This contrasts with non-Hispanic US native
populations, where only 17\% of Black men hold a college degree compared
to 36\% of White men. Hispanic migrants identifying as a race other than
White or Black are the least likely to have a college degree and most
likely to have less than primary education, with 33\% of women and 32\%
of men in this category having less than a primary education.

\subsection{Discussion}\label{sec-discussion}

This study will provide a comprehensive sociodemographic comparison of
older adult Hispanic populations both in their countries of origin and
as immigrants in the United States. The analysis so far reveals
significant variations in education levels marital status and migration
patterns across different Hispanic subgroups. These findings have
important implications for understanding the diverse experiences of
Hispanic older adults and for informing policies related to healthy
aging and caregiving. The ongoing analysis (to be reported in the full
paper) is expanding this analysis to further examine patterns by
immigration cohort, also using multivariate regression to better parse
mechanisms underlying the observed differences.

That Mexican migrants have fewer years of formal education is a finding
that other researchers have come across (Hanson, Orrenius, and Zavodny
2023).

The complexity of migration from Latin America to the United States is
underscored by the region's evolving migration landscape, which has seen
a dramatic increase in intra-regional movement and return migration
since 2010, challenging the traditional narrative of unidirectional
flows to North America and Europe while still maintaining significant
outward migration patterns (Tanco 2023). Additionally, this paper does
not capture return-migrants to Latin America. Lastly, this research
considers migrant populations and host countries pre-pandemic, which is
markedly different than post-pandemic patterns (Hanson, Orrenius, and
Zavodny 2023).

\begin{landscape} 
\newgeometry{layoutwidth=297mm, layoutheight=210mm, left=20mm, right=20mm, top=20mm, bottom=20mm, footskip=15mm}

% country features table

\begin{table}[ht]
\centering
\caption{Latin American Development Indicators} 
\begin{tabular}{l|p{2.5cm}ccc}
Country & GDP Per Capita & Life Expectancy at 60 & Infant Mortality Rate (1950) & Infant Mortality Rate (2019) \\ 
  \hline
Argentina & 14187.5 & 21.59 & 68.00 & 7.75 \\ 
  Belize/British Honduras & 7460 & 21.69 & 89.34 & 12.42 \\ 
  Bolivia & 3686.3 & 19.42 & 182.16 & 28.83 \\ 
  Brazil & 10294.9 & 21.32 & 143.77 & 11.81 \\ 
  Canada & 53431.2 & 24.99 & 41.47 & 4.44 \\ 
  Chile & 17067.8 & 24.57 & 139.00 & 6.55 \\ 
  Colombia & 6947.4 & 22.69 & 97.33 & 9.36 \\ 
  Costa Rica & 16942 & 24.52 & 80.94 & 7.29 \\ 
  Cuba & .. & 21.57 & 87.12 & 4.89 \\ 
  Dominican Republic & 10717.6 & 22.60 & 63.00 & 4.14 \\ 
  Ecuador & 6609.8 & 22.83 & 144.02 & 10.15 \\ 
  El Salvador & 5391.1 & 22.01 & 78.05 & 14.05 \\ 
  Guatemala & 5762.8 & 20.65 & 107.00 & 18.54 \\ 
  Guyana/British Guiana & 20765.4 & 18.01 & 82.80 & 11.40 \\ 
  Haiti & 1705.8 & 16.25 & 237.19 & 53.03 \\ 
  Honduras & 3231.7 & 18.13 & 86.00 & 14.66 \\ 
  Jamaica & 6839.7 & 18.66 & 91.87 & 11.56 \\ 
  Mexico & 13790 & 22.20 & 96.00 & 10.79 \\ 
  Nicaragua & 2612.9 & 24.50 & 55.14 & 12.59 \\ 
  Panama & 18686.4 & 23.98 & 53.02 & 14.27 \\ 
  Paraguay & 6276.4 & 21.01 & 75.69 & 9.45 \\ 
  Peru & 7906.6 & 23.37 & 164.76 & 16.01 \\ 
  Puerto Rico & 36779.1 & 25.32 & 70.53 & 5.31 \\ 
  Trinidad and Tobago & 20016.2 & 21.44 & 74.65 & 21.55 \\ 
  Uruguay & 22797.8 & 21.71 & 62.51 & 5.64 \\ 
  Venezuela & .. & 20.67 & 80.00 & 25.54 \\ 
\end{tabular}
\end{table}

% table 1

\begin{table}[ht]
\centering
\begin{threeparttable}
\caption{Sociodemographic Comparison of Hispanics in the U.S. by Birth Country (2016-20 ACS)} 
\begingroup\small
\begin{tabular}{>{\raggedright\arraybackslash}p{3.2cm}|lcccccc|cccc}
& \multicolumn{7}{c|}{Immigrants} & \multicolumn{4}{c}{Native Born} \\
\cline{2-12}
& \multicolumn{7}{c|}{} & \multicolumn{1}{c}{} & \multicolumn{3}{c}{Non-Hispanic} \\
  \hline
Demographics & Mexico & Puerto Rico$^1$ & Dominican Republic & Cuba & Central America & Latin America$^2$ & Other & Hispanic & Black & White & Other \\ 
  \hline
Age &  &  &  &  &  &  &  &  &  &  &  \\ 
  \multicolumn{1}{>{\raggedleft\arraybackslash}p{1.5cm}|}{\makebox[1.5cm][r]{60 - 69 }}& 0.6 & 0.47 & 0.6 & 0.43 & 0.63 & 0.56 & 0.52 & 0.59 & 0.58 & 0.5 & 0.57 \\ 
  \multicolumn{1}{>{\raggedleft\arraybackslash}p{1.5cm}|}{\makebox[1.5cm][r]{70 - 79 }}& 0.27 & 0.35 & 0.26 & 0.31 & 0.25 & 0.3 & 0.3 & 0.27 & 0.28 & 0.32 & 0.28 \\ 
  \multicolumn{1}{>{\raggedleft\arraybackslash}p{1.5cm}|}{\makebox[1.5cm][r]{80 - 89 }}& 0.1 & 0.15 & 0.11 & 0.21 & 0.1 & 0.12 & 0.14 & 0.12 & 0.11 & 0.14 & 0.12 \\ 
  \multicolumn{1}{>{\raggedleft\arraybackslash}p{1.5cm}|}{\makebox[1.5cm][r]{90 plus }}& 0.02 & 0.03 & 0.03 & 0.05 & 0.02 & 0.03 & 0.04 & 0.03 & 0.03 & 0.04 & 0.03 \\ 
  Education Completed &  &  &  &  &  &  &  &  &  &  &  \\ 
  \multicolumn{1}{>{\raggedleft\arraybackslash}p{3.2cm}|}{\makebox[3.2cm][r]{Less than Primary }}& 0.35 & 0.14 & 0.25 & 0.1 & 0.24 & 0.08 & 0.09 & 0.07 & 0.03 & 0.01 & 0.03 \\ 
  \multicolumn{1}{>{\raggedleft\arraybackslash}p{1.7cm}|}{\makebox[1.7cm][r]{Primary }}& 0.23 & 0.12 & 0.18 & 0.12 & 0.15 & 0.06 & 0.05 & 0.06 & 0.04 & 0.02 & 0.03 \\ 
  \multicolumn{1}{>{\raggedleft\arraybackslash}p{2cm}|}{\makebox[2cm][r]{Secondary }}& 0.34 & 0.57 & 0.44 & 0.51 & 0.45 & 0.54 & 0.45 & 0.64 & 0.69 & 0.59 & 0.57 \\ 
  \multicolumn{1}{>{\raggedleft\arraybackslash}p{2cm}|}{\makebox[2cm][r]{University }}& 0.08 & 0.18 & 0.13 & 0.27 & 0.15 & 0.31 & 0.42 & 0.23 & 0.25 & 0.39 & 0.38 \\ 
  Household$^3$ &  &  &  &  &  &  &  &  &  &  &  \\ 
  \multicolumn{1}{>{\raggedleft\arraybackslash}p{2.7cm}|}{\makebox[2.7cm][r]{Household Size }}& 3.32 & 2.31 & 2.87 & 2.46 & 3.14 & 2.73 & 2.65 & 2.37 & 2.06 & 1.95 & 2.27 \\ 
  \multicolumn{1}{>{\raggedleft\arraybackslash}p{2.2cm}|}{\makebox[2.2cm][r]{Lives Alone }}& 0.16 & 0.32 & 0.24 & 0.27 & 0.18 & 0.21 & 0.21 & 0.28 & 0.39 & 0.29 & 0.31 \\ 
  \multicolumn{1}{>{\raggedleft\arraybackslash}p{2.9cm}|}{\makebox[2.9cm][r]{Lives with Child }}& 0.51 & 0.31 & 0.45 & 0.32 & 0.48 & 0.4 & 0.35 & 0.28 & 0.24 & 0.14 & 0.25 \\ 
  \multicolumn{1}{>{\raggedleft\arraybackslash}p{3.4cm}|}{\makebox[3.4cm][r]{Married/Cohabiting }}& 0.56 & 0.42 & 0.41 & 0.44 & 0.46 & 0.52 & 0.59 & 0.49 & 0.35 & 0.59 & 0.48 \\ 
  Age Migrated &  &  &  &  &  &  &  &  &  &  &  \\ 
  \multicolumn{1}{>{\raggedleft\arraybackslash}p{2.4cm}|}{\makebox[2.4cm][r]{Less than 15 }}& 0.11 & 0.29 & 0.06 & 0.17 & 0.06 & 0.07 & 0.13 & - & - & - & - \\ 
  \multicolumn{1}{>{\raggedleft\arraybackslash}p{1.6cm}|}{\makebox[1.6cm][r]{15 - 23 }}& 0.24 & 0.26 & 0.15 & 0.12 & 0.17 & 0.14 & 0.14 & - & - & - & - \\ 
  \multicolumn{1}{>{\raggedleft\arraybackslash}p{1.6cm}|}{\makebox[1.6cm][r]{24 - 49 }}& 0.48 & 0.27 & 0.53 & 0.44 & 0.59 & 0.54 & 0.5 & - & - & - & - \\ 
  \multicolumn{1}{>{\raggedleft\arraybackslash}p{2.6cm}|}{\makebox[2.6cm][r]{50 and Above }}& 0.13 & 0.16 & 0.23 & 0.25 & 0.15 & 0.22 & 0.2 & - & - & - & - \\ 
  Migration Cohort &  &  &  &  &  &  &  &  &  &  &  \\ 
  \multicolumn{1}{>{\raggedleft\arraybackslash}p{2.3cm}|}{\makebox[2.3cm][r]{Before 1965 }}& 0.15 & 0.41 & 0.08 & 0.25 & 0.09 & 0.11 & 0.21 & - & - & - & - \\ 
  \multicolumn{1}{>{\raggedleft\arraybackslash}p{2.2cm}|}{\makebox[2.2cm][r]{1965 - 1979 }}& 0.45 & 0.28 & 0.33 & 0.35 & 0.37 & 0.34 & 0.31 & - & - & - & - \\ 
  \multicolumn{1}{>{\raggedleft\arraybackslash}p{2.2cm}|}{\makebox[2.2cm][r]{1980 - 1999 }}& 0.33 & 0.18 & 0.43 & 0.28 & 0.46 & 0.36 & 0.35 & - & - & - & - \\ 
  \multicolumn{1}{>{\raggedleft\arraybackslash}p{2.1cm}|}{\makebox[2.1cm][r]{After 1999 }}& 0.12 & 0.14 & 0.2 & 0.21 & 0.14 & 0.23 & 0.17 & - & - & - & - \\ 
  Acculturation &  &  &  &  &  &  &  &  &  &  &  \\ 
  \multicolumn{1}{>{\raggedleft\arraybackslash}p{1.6cm}|}{\makebox[1.6cm][r]{Citizen }}& 0.54 & - & 0.68 & 0.8 & 0.66 & 0.71 & 0.74 & - & - & - & - \\ 
  \multicolumn{1}{>{\raggedleft\arraybackslash}p{3cm}|}{\makebox[3cm][r]{English Speakers }}& 0.73 & 0.91 & 0.69 & 0.76 & 0.82 & 0.87 & 0.93 & 0.99 & 1 & 1 & 1 \\ 
    \hline
  N & 79,658 & 23,021 & 8,980 & 21,242 & 20,064 & 25,003 & 352,960 & 120,724 & 313,063 & 3,165,675 & 94,162 \\ 
   \hline
\end{tabular}
\endgroup
\begin{tablenotes}
\item[1] A Puerto Rican migrant is defined as someone who moves from Puerto Rico to mainland US.
\item[2] The Latin American category excludes our four countries of interest.
\item[3] Household size is assessed as how many "own family" members in the household. "Lives with Child" are those who live with at least one adult-child.
\end{tablenotes}
\end{threeparttable}
\end{table}

% table 2

\begin{table}[ht]
\centering
\caption{Summary Statistics by Country and Sex For Hispanics in Their Native Countries} 
\begingroup\small
\begin{tabular}{l|>{\raggedright\arraybackslash}p{3.2cm}|ccccccccc}
  \hline
Gender & Demographics & Mexicans & US Mexican-born & Puerto Ricans & US Puerto-Rican-born & Dominicans & US Dominican-born & Cubans & US Cuban-born \\ 
  \hline
Female & Age &  &  &  &  &  &  &  &  \\ 
   & \multicolumn{1}{>{\raggedleft\arraybackslash}p{1.5cm}|}{\makebox[1.5cm][r]{60 - 69 }} & 0.53 & 0.57 & 0.5 & 0.57 & 0.52 & 0.59 & 0.5 & 0.37 \\ 
   & \multicolumn{1}{>{\raggedleft\arraybackslash}p{1.5cm}|}{\makebox[1.5cm][r]{70 - 79 }} & 0.31 & 0.29 & 0.32 & 0.28 & 0.31 & 0.27 & 0.31 & 0.36 \\ 
   & \multicolumn{1}{>{\raggedleft\arraybackslash}p{1.5cm}|}{\makebox[1.5cm][r]{80 - 89 }} & 0.13 & 0.11 & 0.15 & 0.13 & 0.14 & 0.11 & 0.15 & 0.23 \\ 
   & \multicolumn{1}{>{\raggedleft\arraybackslash}p{1.5cm}|}{\makebox[1.5cm][r]{90 plus }} & 0.03 & 0.03 & 0.03 & 0.03 & 0.04 & 0.03 & 0.04 & 0.04 \\ 
   & Education Completed &  &  &  &  &  &  &  &  \\ 
   & \multicolumn{1}{>{\raggedleft\arraybackslash}p{3.2cm}|}{\makebox[3.2cm][r]{Less than Primary }} & 0.61 & 0.43 & 0.26 & 0.18 & 0.72 & 0.34 & 0.27 & 0.12 \\ 
   & \multicolumn{1}{>{\raggedleft\arraybackslash}p{1.7cm}|}{\makebox[1.7cm][r]{Primary }} & 0.29 & 0.32 & 0.25 & 0.3 & 0.18 & 0.31 & 0.49 & 0.26 \\ 
   & \multicolumn{1}{>{\raggedleft\arraybackslash}p{2cm}|}{\makebox[2cm][r]{Secondary }} & 0.07 & 0.23 & 0.36 & 0.44 & 0.07 & 0.3 & 0.17 & 0.47 \\ 
   & \multicolumn{1}{>{\raggedleft\arraybackslash}p{2cm}|}{\makebox[2cm][r]{University }} & 0.03 & 0.03 & 0.13 & 0.09 & 0.04 & 0.06 & 0.06 & 0.15 \\ 
   & Household &  &  &  &  &  &  &  &  \\ 
   & \multicolumn{1}{>{\raggedleft\arraybackslash}p{2.7cm}|}{\makebox[2.7cm][r]{Household Size }} & 3.56 & 3.57 & 2.23 & 2.37 & 3.55 & 3.11 & 3.01 & 2.48 \\ 
   & \multicolumn{1}{>{\raggedleft\arraybackslash}p{2.2cm}|}{\makebox[2.2cm][r]{Lives Alone }} & 0.13 & 0.16 & 0.27 & 0.32 & 0.12 & 0.25 & 0.14 & 0.29 \\ 
   & \multicolumn{1}{>{\raggedleft\arraybackslash}p{2.9cm}|}{\makebox[2.9cm][r]{Lives with Child }} & 0.58 & 0.42 & 0.29 & 0.31 & 0.55 & 0.45 & 0.52 & 0.3 \\ 
   & \multicolumn{1}{>{\raggedleft\arraybackslash}p{3.4cm}|}{\makebox[3.4cm][r]{Married/Cohabiting }} & 0.46 & 0.47 & 0.42 & 0.35 & 0.39 & 0.3 & 0.46 & 0.35 \\ 
    \hline
   & N & 624788 & 6055 & 5267 & 2148 & 44053 & 728 & 107811 & 2211 \\ 
    \hline
  Male & Age &  &  &  &  &  &  &  &  \\ 
   & \multicolumn{1}{>{\raggedleft\arraybackslash}p{1.5cm}|}{\makebox[1.5cm][r]{60 - 69 }} & 0.54 & 0.63 & 0.54 & 0.59 & 0.54 & 0.6 & 0.52 & 0.43 \\ 
   & \multicolumn{1}{>{\raggedleft\arraybackslash}p{1.5cm}|}{\makebox[1.5cm][r]{70 - 79 }} & 0.31 & 0.27 & 0.32 & 0.3 & 0.3 & 0.3 & 0.32 & 0.35 \\ 
   & \multicolumn{1}{>{\raggedleft\arraybackslash}p{1.5cm}|}{\makebox[1.5cm][r]{80 - 89 }} & 0.12 & 0.09 & 0.13 & 0.09 & 0.12 & 0.09 & 0.13 & 0.2 \\ 
   & \multicolumn{1}{>{\raggedleft\arraybackslash}p{1.5cm}|}{\makebox[1.5cm][r]{90 plus }} & 0.03 & 0.02 & 0.02 & 0.02 & 0.03 & 0.01 & 0.03 & 0.02 \\ 
   & Education Completed &  &  &  &  &  &  &  &  \\ 
   & \multicolumn{1}{>{\raggedleft\arraybackslash}p{3.2cm}|}{\makebox[3.2cm][r]{Less than Primary }} & 0.56 & 0.44 & 0.21 & 0.17 & 0.67 & 0.26 & 0.21 & 0.12 \\ 
   & \multicolumn{1}{>{\raggedleft\arraybackslash}p{1.7cm}|}{\makebox[1.7cm][r]{Primary }} & 0.29 & 0.3 & 0.27 & 0.29 & 0.21 & 0.3 & 0.48 & 0.24 \\ 
   & \multicolumn{1}{>{\raggedleft\arraybackslash}p{2cm}|}{\makebox[2cm][r]{Secondary }} & 0.07 & 0.21 & 0.38 & 0.44 & 0.07 & 0.33 & 0.23 & 0.46 \\ 
   & \multicolumn{1}{>{\raggedleft\arraybackslash}p{2cm}|}{\makebox[2cm][r]{University }} & 0.08 & 0.05 & 0.14 & 0.1 & 0.05 & 0.11 & 0.08 & 0.17 \\ 
   & Household &  &  &  &  &  &  &  &  \\ 
   & \multicolumn{1}{>{\raggedleft\arraybackslash}p{2.7cm}|}{\makebox[2.7cm][r]{Household Size }} & 3.7 & 3.68 & 2.4 & 2.55 & 3.56 & 3.5 & 2.95 & 2.53 \\ 
   & \multicolumn{1}{>{\raggedleft\arraybackslash}p{2.2cm}|}{\makebox[2.2cm][r]{Lives Alone }} & 0.11 & 0.13 & 0.18 & 0.25 & 0.16 & 0.16 & 0.17 & 0.22 \\ 
   & \multicolumn{1}{>{\raggedleft\arraybackslash}p{2.9cm}|}{\makebox[2.9cm][r]{Lives with Child }} & 0.56 & 0.34 & 0.24 & 0.25 & 0.5 & 0.39 & 0.43 & 0.2 \\ 
   & \multicolumn{1}{>{\raggedleft\arraybackslash}p{3.4cm}|}{\makebox[3.4cm][r]{Married/Cohabiting }} & 0.74 & 0.75 & 0.67 & 0.61 & 0.68 & 0.67 & 0.67 & 0.66 \\ 
    \hline
   & N & 572708 & 5038 & 3989 & 1614 & 41426 & 427 & 96337 & 1675 \\ 
   \hline
\end{tabular}
\endgroup
\end{table}
\end{landscape}

\newpage

\subsection{Appendix}\label{sec-Appendix}

\newpage

\begin{landscape} 
\newgeometry{layoutwidth=297mm, layoutheight=210mm, left=20mm, right=20mm, top=20mm, bottom=20mm, footskip=15mm}

%table 1, females only

\begin{table}[ht]
\centering
\caption{Sociodemographic Comparison of Hispanics in the U.S. by Birth Country (2016-20 ACS): Females} 
\begingroup\small
\begin{tabular}{>{\raggedright\arraybackslash}p{3.2cm}|lcccccc|cccc}
  \hline
Demographics & Mexico & Puerto Rico & Dominican Republic & Cuba & Central America & Latin America & Other Countries & Hispanic & Black & White & Other \\ 
  \hline
Age &  &  &  &  &  &  &  &  &  &  &  \\ 
  \multicolumn{1}{>{\raggedleft\arraybackslash}p{1.5cm}|}{\makebox[1.5cm][r]{60 - 69 }}& 0.57 & 0.45 & 0.58 & 0.4 & 0.63 & 0.56 & 0.52 & 0.59 & 0.55 & 0.48 & 0.56 \\ 
  \multicolumn{1}{>{\raggedleft\arraybackslash}p{1.5cm}|}{\makebox[1.5cm][r]{70 - 79 }}& 0.28 & 0.35 & 0.27 & 0.31 & 0.25 & 0.3 & 0.3 & 0.27 & 0.28 & 0.31 & 0.27 \\ 
  \multicolumn{1}{>{\raggedleft\arraybackslash}p{1.5cm}|}{\makebox[1.5cm][r]{80 - 89 }}& 0.12 & 0.16 & 0.12 & 0.23 & 0.1 & 0.12 & 0.14 & 0.12 & 0.13 & 0.15 & 0.13 \\ 
  \multicolumn{1}{>{\raggedleft\arraybackslash}p{1.5cm}|}{\makebox[1.5cm][r]{90 plus }}& 0.03 & 0.04 & 0.03 & 0.06 & 0.02 & 0.03 & 0.04 & 0.03 & 0.04 & 0.05 & 0.04 \\ 
  Education Completed &  &  &  &  &  &  &  &  &  &  &  \\ 
  \multicolumn{1}{>{\raggedleft\arraybackslash}p{3.2cm}|}{\makebox[3.2cm][r]{Less than Primary }}& 0.36 & 0.14 & 0.26 & 0.1 & 0.24 & 0.08 & 0.09 & 0.07 & 0.02 & 0.01 & 0.03 \\ 
  \multicolumn{1}{>{\raggedleft\arraybackslash}p{1.7cm}|}{\makebox[1.7cm][r]{Primary }}& 0.23 & 0.12 & 0.19 & 0.13 & 0.15 & 0.06 & 0.05 & 0.06 & 0.03 & 0.02 & 0.03 \\ 
  \multicolumn{1}{>{\raggedleft\arraybackslash}p{2cm}|}{\makebox[2cm][r]{Secondary }}& 0.34 & 0.55 & 0.43 & 0.51 & 0.45 & 0.54 & 0.45 & 0.64 & 0.68 & 0.62 & 0.58 \\ 
  \multicolumn{1}{>{\raggedleft\arraybackslash}p{2cm}|}{\makebox[2cm][r]{University }}& 0.07 & 0.19 & 0.12 & 0.26 & 0.15 & 0.31 & 0.42 & 0.23 & 0.26 & 0.36 & 0.37 \\ 
  Household &  &  &  &  &  &  &  &  &  &  &  \\ 
  \multicolumn{1}{>{\raggedleft\arraybackslash}p{2.7cm}|}{\makebox[2.7cm][r]{Household Size }}& 3.29 & 2.25 & 2.78 & 2.44 & 3.14 & 2.73 & 2.65 & 2.37 & 2.05 & 1.89 & 2.26 \\ 
  \multicolumn{1}{>{\raggedleft\arraybackslash}p{2.2cm}|}{\makebox[2.2cm][r]{Lives Alone }}& 0.17 & 0.35 & 0.26 & 0.3 & 0.18 & 0.21 & 0.21 & 0.28 & 0.41 & 0.35 & 0.33 \\ 
  \multicolumn{1}{>{\raggedleft\arraybackslash}p{2.9cm}|}{\makebox[2.9cm][r]{Lives with Child }}& 0.51 & 0.33 & 0.46 & 0.35 & 0.48 & 0.4 & 0.35 & 0.28 & 0.27 & 0.15 & 0.27 \\ 
  \multicolumn{1}{>{\raggedleft\arraybackslash}p{3.4cm}|}{\makebox[3.4cm][r]{Married/Cohabiting }}& 0.47 & 0.32 & 0.29 & 0.33 & 0.46 & 0.52 & 0.59 & 0.49 & 0.27 & 0.51 & 0.41 \\ 
  Age Migrated &  &  &  &  &  &  &  &  &  &  &  \\ 
  \multicolumn{1}{>{\raggedleft\arraybackslash}p{2.4cm}|}{\makebox[2.4cm][r]{Less than 15 }}& 0.11 & 0.28 & 0.06 & 0.15 & 0.06 & 0.07 & 0.13 & - & - & - & - \\ 
  \multicolumn{1}{>{\raggedleft\arraybackslash}p{1.6cm}|}{\makebox[1.6cm][r]{15 - 23 }}& 0.28 & 0.29 & 0.19 & 0.16 & 0.23 & 0.2 & 0.2 & - & - & - & - \\ 
  \multicolumn{1}{>{\raggedleft\arraybackslash}p{1.6cm}|}{\makebox[1.6cm][r]{24 - 49 }}& 0.45 & 0.26 & 0.51 & 0.41 & 0.56 & 0.51 & 0.47 & - & - & - & - \\ 
  \multicolumn{1}{>{\raggedleft\arraybackslash}p{2.6cm}|}{\makebox[2.6cm][r]{50 and Above }}& 0.16 & 0.17 & 0.25 & 0.28 & 0.15 & 0.22 & 0.2 & 1 & 1 & 1 & 1 \\ 
  Migration Cohort &  &  &  &  &  &  &  &  &  &  &  \\ 
  \multicolumn{1}{>{\raggedleft\arraybackslash}p{2.3cm}|}{\makebox[2.3cm][r]{Before 1965 }}& 0.16 & 0.41 & 0.09 & 0.24 & 0.09 & 0.11 & 0.21 & - & - & - & - \\ 
  \multicolumn{1}{>{\raggedleft\arraybackslash}p{2.2cm}|}{\makebox[2.2cm][r]{1965 - 1979 }}& 0.42 & 0.27 & 0.31 & 0.36 & 0.37 & 0.34 & 0.31 & - & - & - & - \\ 
  \multicolumn{1}{>{\raggedleft\arraybackslash}p{2.2cm}|}{\makebox[2.2cm][r]{1980 - 1999 }}& 0.33 & 0.19 & 0.43 & 0.23 & 0.46 & 0.36 & 0.35 & - & - & - & - \\ 
  \multicolumn{1}{>{\raggedleft\arraybackslash}p{2.1cm}|}{\makebox[2.1cm][r]{After 1999 }}& 0.14 & 0.15 & 0.21 & 0.23 & 0.14 & 0.23 & 0.17 & - & - & - & - \\ 
  Acculturation &  &  &  &  &  &  &  &  &  &  &  \\ 
  \multicolumn{1}{>{\raggedleft\arraybackslash}p{1.6cm}|}{\makebox[1.6cm][r]{Citizen }}& 0.55 & - & 0.7 & 0.82 & 0.66 & 0.71 & 0.74 & - & - & - & - \\ 
  \multicolumn{1}{>{\raggedleft\arraybackslash}p{3cm}|}{\makebox[3cm][r]{English Speakers }}& 0.68 & 0.9 & 0.65 & 0.72 & 0.82 & 0.87 & 0.93 & 0.99 & 1 & 1 & 1 \\ 
  N & 42189 & 13298 & 5390 & 11852 & 20064 & 25003 & 352960 & 120724 & 182312 & 1695106 & 51222 \\ 
   \hline
\end{tabular}
\endgroup
\end{table}

% table 1, males only

\begin{table}[ht]
\centering
\caption{Sociodemographic Comparison of Hispanics in the U.S. by Birth Country (2016-20 ACS): Males} 
\begingroup\small
\begin{tabular}{>{\raggedright\arraybackslash}p{3.2cm}|lcccccc|cccc}
  \hline
Demographics & Mexico & Puerto Rico & Dominican Republic & Cuba & Central America & Latin America & Other Countries & Hispanic & Black & White & Other \\ 
  \hline
Age &  &  &  &  &  &  &  &  &  &  &  \\ 
  \multicolumn{1}{>{\raggedleft\arraybackslash}p{1.5cm}|}{\makebox[1.5cm][r]{60 - 69 }}& 0.63 & 0.5 & 0.63 & 0.47 & 0.63 & 0.56 & 0.52 & 0.59 & 0.61 & 0.53 & 0.59 \\ 
  \multicolumn{1}{>{\raggedleft\arraybackslash}p{1.5cm}|}{\makebox[1.5cm][r]{70 - 79 }}& 0.26 & 0.35 & 0.26 & 0.31 & 0.25 & 0.3 & 0.3 & 0.27 & 0.28 & 0.32 & 0.28 \\ 
  \multicolumn{1}{>{\raggedleft\arraybackslash}p{1.5cm}|}{\makebox[1.5cm][r]{80 - 89 }}& 0.09 & 0.13 & 0.1 & 0.18 & 0.1 & 0.12 & 0.14 & 0.12 & 0.1 & 0.13 & 0.11 \\ 
  \multicolumn{1}{>{\raggedleft\arraybackslash}p{1.5cm}|}{\makebox[1.5cm][r]{90 plus }}& 0.02 & 0.02 & 0.02 & 0.04 & 0.02 & 0.03 & 0.04 & 0.03 & 0.02 & 0.03 & 0.02 \\ 
  Education Completed &  &  &  &  &  &  &  &  &  &  &  \\ 
  \multicolumn{1}{>{\raggedleft\arraybackslash}p{3.2cm}|}{\makebox[3.2cm][r]{Less than Primary }}& 0.35 & 0.14 & 0.23 & 0.09 & 0.24 & 0.08 & 0.09 & 0.07 & 0.03 & 0.01 & 0.03 \\ 
  \multicolumn{1}{>{\raggedleft\arraybackslash}p{1.7cm}|}{\makebox[1.7cm][r]{Primary }}& 0.23 & 0.11 & 0.17 & 0.11 & 0.15 & 0.06 & 0.05 & 0.06 & 0.04 & 0.02 & 0.03 \\ 
  \multicolumn{1}{>{\raggedleft\arraybackslash}p{2cm}|}{\makebox[2cm][r]{Secondary }}& 0.34 & 0.58 & 0.46 & 0.5 & 0.45 & 0.54 & 0.45 & 0.64 & 0.7 & 0.55 & 0.55 \\ 
  \multicolumn{1}{>{\raggedleft\arraybackslash}p{2cm}|}{\makebox[2cm][r]{University }}& 0.09 & 0.17 & 0.14 & 0.29 & 0.15 & 0.31 & 0.42 & 0.23 & 0.23 & 0.42 & 0.4 \\ 
  Household &  &  &  &  &  &  &  &  &  &  &  \\ 
  \multicolumn{1}{>{\raggedleft\arraybackslash}p{2.7cm}|}{\makebox[2.7cm][r]{Household Size }}& 3.35 & 2.4 & 3.01 & 2.49 & 3.14 & 2.73 & 2.65 & 2.37 & 2.07 & 2.02 & 2.28 \\ 
  \multicolumn{1}{>{\raggedleft\arraybackslash}p{2.2cm}|}{\makebox[2.2cm][r]{Lives Alone }}& 0.15 & 0.26 & 0.2 & 0.23 & 0.18 & 0.21 & 0.21 & 0.28 & 0.36 & 0.23 & 0.28 \\ 
  \multicolumn{1}{>{\raggedleft\arraybackslash}p{2.9cm}|}{\makebox[2.9cm][r]{Lives with Child }}& 0.5 & 0.28 & 0.44 & 0.28 & 0.48 & 0.4 & 0.35 & 0.28 & 0.19 & 0.14 & 0.22 \\ 
  \multicolumn{1}{>{\raggedleft\arraybackslash}p{3.4cm}|}{\makebox[3.4cm][r]{Married/Cohabiting }}& 0.67 & 0.56 & 0.59 & 0.59 & 0.46 & 0.52 & 0.59 & 0.49 & 0.47 & 0.68 & 0.58 \\ 
  Age Migrated &  &  &  &  &  &  &  &  &  &  &  \\ 
  \multicolumn{1}{>{\raggedleft\arraybackslash}p{2.4cm}|}{\makebox[2.4cm][r]{Less than 15 }}& 0.12 & 0.3 & 0.07 & 0.19 & 0.06 & 0.07 & 0.13 & - & - & - & - \\ 
  \multicolumn{1}{>{\raggedleft\arraybackslash}p{1.6cm}|}{\makebox[1.6cm][r]{15 - 23 }}& 0.34 & 0.31 & 0.19 & 0.15 & 0.23 & 0.2 & 0.2 & - & - & - & - \\ 
  \multicolumn{1}{>{\raggedleft\arraybackslash}p{1.6cm}|}{\makebox[1.6cm][r]{24 - 49 }}& 0.44 & 0.24 & 0.52 & 0.44 & 0.56 & 0.51 & 0.47 & - & - & - & - \\ 
  \multicolumn{1}{>{\raggedleft\arraybackslash}p{2.6cm}|}{\makebox[2.6cm][r]{50 and Above }}& 0.1 & 0.15 & 0.21 & 0.22 & 0.15 & 0.22 & 0.2 & 1 & 1 & 1 & 1 \\ 
  Migration Cohort &  &  &  &  &  &  &  &  &  &  &  \\ 
  \multicolumn{1}{>{\raggedleft\arraybackslash}p{2.3cm}|}{\makebox[2.3cm][r]{Before 1965 }}& 0.15 & 0.42 & 0.07 & 0.26 & 0.09 & 0.11 & 0.21 & - & - & - & - \\ 
  \multicolumn{1}{>{\raggedleft\arraybackslash}p{2.2cm}|}{\makebox[2.2cm][r]{1965 - 1979 }}& 0.48 & 0.3 & 0.34 & 0.35 & 0.37 & 0.34 & 0.31 & - & - & - & - \\ 
  \multicolumn{1}{>{\raggedleft\arraybackslash}p{2.2cm}|}{\makebox[2.2cm][r]{1980 - 1999 }}& 0.32 & 0.18 & 0.44 & 0.33 & 0.46 & 0.36 & 0.35 & - & - & - & - \\ 
  \multicolumn{1}{>{\raggedleft\arraybackslash}p{2.1cm}|}{\makebox[2.1cm][r]{After 1999 }}& 0.1 & 0.13 & 0.19 & 0.19 & 0.14 & 0.23 & 0.17 & - & - & - & - \\ 
  Acculturation &  &  &  &  &  &  &  &  &  &  &  \\ 
  \multicolumn{1}{>{\raggedleft\arraybackslash}p{1.6cm}|}{\makebox[1.6cm][r]{Citizen }}& 0.53 & - & 0.66 & 0.77 & 0.66 & 0.71 & 0.74 & - & - & - & - \\ 
  \multicolumn{1}{>{\raggedleft\arraybackslash}p{3cm}|}{\makebox[3cm][r]{English Speakers }}& 0.78 & 0.94 & 0.74 & 0.8 & 0.82 & 0.87 & 0.93 & 0.99 & 1 & 1 & 1 \\ 
  N & 37469 & 9723 & 3590 & 9390 & 20064 & 25003 & 352960 & 120724 & 130751 & 1470569 & 42940 \\ 
   \hline
\end{tabular}
\endgroup
\end{table}

% table 1, using 2010 data

\begin{table}[ht]
\centering
\caption{Sociodemographic Comparison of Hispanics in the U.S. by Birth Country (2006-10 ACS)} 
\begingroup\small
\begin{tabular}{>{\raggedright\arraybackslash}p{3.2cm}|lcccccc|cccc}
  \hline
Demographics & Mexico & Puerto Rico & Dominican Republic & Cuba & Central America & Latin America & Other Countries & Hispanic & Black & White & Other \\ 
  \hline
Age &  &  &  &  &  &  &  &  &  &  &  \\ 
  \multicolumn{1}{>{\raggedleft\arraybackslash}p{1.5cm}|}{\makebox[1.5cm][r]{60 - 69 }}& 0.6 & 0.57 & 0.56 & 0.41 & 0.6 & 0.58 & 0.5 & 0.53 & 0.54 & 0.49 & 0.54 \\ 
  \multicolumn{1}{>{\raggedleft\arraybackslash}p{1.5cm}|}{\makebox[1.5cm][r]{70 - 79 }}& 0.28 & 0.29 & 0.3 & 0.36 & 0.27 & 0.29 & 0.31 & 0.31 & 0.3 & 0.3 & 0.28 \\ 
  \multicolumn{1}{>{\raggedleft\arraybackslash}p{1.5cm}|}{\makebox[1.5cm][r]{80 - 89 }}& 0.1 & 0.12 & 0.11 & 0.19 & 0.11 & 0.11 & 0.16 & 0.14 & 0.14 & 0.18 & 0.15 \\ 
  \multicolumn{1}{>{\raggedleft\arraybackslash}p{1.5cm}|}{\makebox[1.5cm][r]{90 plus }}& 0.02 & 0.02 & 0.02 & 0.03 & 0.02 & 0.02 & 0.03 & 0.02 & 0.03 & 0.04 & 0.03 \\ 
  Education Completed &  &  &  &  &  &  &  &  &  &  &  \\ 
  \multicolumn{1}{>{\raggedleft\arraybackslash}p{3.2cm}|}{\makebox[3.2cm][r]{Less than Primary }}& 0.62 & 0.25 & 0.38 & 0.24 & 0.38 & 0.16 & 0.13 & 0.16 & 0.08 & 0.02 & 0.06 \\ 
  \multicolumn{1}{>{\raggedleft\arraybackslash}p{1.7cm}|}{\makebox[1.7cm][r]{Primary }}& 0.06 & 0.11 & 0.16 & 0.08 & 0.06 & 0.05 & 0.05 & 0.09 & 0.07 & 0.04 & 0.05 \\ 
  \multicolumn{1}{>{\raggedleft\arraybackslash}p{2cm}|}{\makebox[2cm][r]{Secondary }}& 0.26 & 0.52 & 0.36 & 0.47 & 0.41 & 0.55 & 0.48 & 0.6 & 0.68 & 0.65 & 0.62 \\ 
  \multicolumn{1}{>{\raggedleft\arraybackslash}p{2cm}|}{\makebox[2cm][r]{University }}& 0.06 & 0.12 & 0.09 & 0.21 & 0.15 & 0.24 & 0.34 & 0.15 & 0.17 & 0.29 & 0.27 \\ 
  Household &  &  &  &  &  &  &  &  &  &  &  \\ 
  \multicolumn{1}{>{\raggedleft\arraybackslash}p{2.7cm}|}{\makebox[2.7cm][r]{Household Size }}& 3.48 & 2.33 & 3.06 & 2.44 & 3.21 & 2.79 & 2.55 & 2.38 & 2.12 & 1.9 & 2.23 \\ 
  \multicolumn{1}{>{\raggedleft\arraybackslash}p{2.2cm}|}{\makebox[2.2cm][r]{Lives Alone }}& 0.15 & 0.3 & 0.22 & 0.25 & 0.19 & 0.2 & 0.23 & 0.26 & 0.37 & 0.31 & 0.3 \\ 
  \multicolumn{1}{>{\raggedleft\arraybackslash}p{2.9cm}|}{\makebox[2.9cm][r]{Lives with Child }}& 0.53 & 0.31 & 0.49 & 0.29 & 0.48 & 0.42 & 0.32 & 0.28 & 0.25 & 0.13 & 0.24 \\ 
  \multicolumn{1}{>{\raggedleft\arraybackslash}p{3.4cm}|}{\makebox[3.4cm][r]{Married/Cohabiting }}& 0.54 & 0.44 & 0.37 & 0.5 & 0.4 & 0.51 & 0.57 & 0.52 & 0.36 & 0.58 & 0.5 \\ 
  Age Migrated &  &  &  &  &  &  &  &  &  &  &  \\ 
  \multicolumn{1}{>{\raggedleft\arraybackslash}p{2.4cm}|}{\makebox[2.4cm][r]{Less than 15 }}& 0.09 & 0.26 & 0.03 & 0.05 & 0.03 & 0.03 & 0.11 & - & - & - & - \\ 
  \multicolumn{1}{>{\raggedleft\arraybackslash}p{1.6cm}|}{\makebox[1.6cm][r]{15 - 23 }}& 0.21 & 0.36 & 0.15 & 0.15 & 0.16 & 0.17 & 0.2 & - & - & - & - \\ 
  \multicolumn{1}{>{\raggedleft\arraybackslash}p{1.6cm}|}{\makebox[1.6cm][r]{24 - 49 }}& 0.48 & 0.25 & 0.5 & 0.57 & 0.54 & 0.52 & 0.45 & - & - & - & - \\ 
  \multicolumn{1}{>{\raggedleft\arraybackslash}p{2.6cm}|}{\makebox[2.6cm][r]{50 and Above }}& 0.21 & 0.13 & 0.32 & 0.23 & 0.27 & 0.29 & 0.24 & 1 & 1 & 1 & 1 \\ 
  Migration Cohort &  &  &  &  &  &  &  &  &  &  &  \\ 
  \multicolumn{1}{>{\raggedleft\arraybackslash}p{2.3cm}|}{\makebox[2.3cm][r]{Before 1965 }}& 0.29 & 0.64 & 0.18 & 0.32 & 0.18 & 0.19 & 0.35 & - & - & - & - \\ 
  \multicolumn{1}{>{\raggedleft\arraybackslash}p{2.2cm}|}{\makebox[2.2cm][r]{1965 - 1979 }}& 0.38 & 0.19 & 0.36 & 0.41 & 0.37 & 0.4 & 0.31 & - & - & - & - \\ 
  \multicolumn{1}{>{\raggedleft\arraybackslash}p{2.2cm}|}{\makebox[2.2cm][r]{1980 - 1999 }}& 0.26 & 0.12 & 0.37 & 0.26 & 0.42 & 0.3 & 0.28 & - & - & - & - \\ 
  \multicolumn{1}{>{\raggedleft\arraybackslash}p{2.1cm}|}{\makebox[2.1cm][r]{After 1999 }}& 0.09 & 0.06 & 0.11 & 0.09 & 0.09 & 0.14 & 0.08 & - & - & - & - \\ 
  Acculturation &  &  &  &  &  &  &  &  &  &  &  \\ 
  \multicolumn{1}{>{\raggedleft\arraybackslash}p{1.6cm}|}{\makebox[1.6cm][r]{Citizen }}& 0.5 & - & 0.59 & 0.77 & 0.58 & 0.66 & 0.73 & - & - & - & - \\ 
  \multicolumn{1}{>{\raggedleft\arraybackslash}p{3cm}|}{\makebox[3cm][r]{English Speakers }}& 0.63 & 0.91 & 0.61 & 0.73 & 0.73 & 0.82 & 0.91 & 0.98 & 1 & 1 & 0.99 \\ 
  N & 46202 & 17403 & 4750 & 17697 & 10241 & 14534 & 243972 & 73429 & 226793 & 2515298 & 54477 \\ 
   \hline
\end{tabular}
\endgroup
\end{table}

% table 1, focusing purely on those 25 and above

\begin{table}[ht]
\centering
\caption{Sociodemographic Comparison of Hispanics in the U.S. by Birth Country (2016-20 ACS): Migrated After Age 24} 
\begingroup\small
\begin{tabular}{>{\raggedright\arraybackslash}p{3.2cm}|lcccccc|cccc}
  \hline
Demographics & Mexico & Puerto Rico & Dominican Republic & Cuba & Central America & Latin America & Other Countries & Hispanic & Black & White & Other \\ 
  \hline
Age &  &  &  &  &  &  &  &  &  &  &  \\ 
  \multicolumn{1}{>{\raggedleft\arraybackslash}p{1.5cm}|}{\makebox[1.5cm][r]{60 - 69 }}& 0.52 & 0.46 & 0.56 & 0.34 & 0.59 & 0.53 & 0.49 & 0.59 & 0.58 & 0.5 & 0.57 \\ 
  \multicolumn{1}{>{\raggedleft\arraybackslash}p{1.5cm}|}{\makebox[1.5cm][r]{70 - 79 }}& 0.31 & 0.31 & 0.27 & 0.32 & 0.27 & 0.29 & 0.31 & 0.27 & 0.28 & 0.32 & 0.28 \\ 
  \multicolumn{1}{>{\raggedleft\arraybackslash}p{1.5cm}|}{\makebox[1.5cm][r]{80 - 89 }}& 0.14 & 0.17 & 0.13 & 0.27 & 0.12 & 0.15 & 0.16 & 0.12 & 0.11 & 0.14 & 0.12 \\ 
  \multicolumn{1}{>{\raggedleft\arraybackslash}p{1.5cm}|}{\makebox[1.5cm][r]{90 plus }}& 0.03 & 0.05 & 0.03 & 0.07 & 0.03 & 0.03 & 0.04 & 0.03 & 0.03 & 0.04 & 0.03 \\ 
  Education Completed &  &  &  &  &  &  &  &  &  &  &  \\ 
  \multicolumn{1}{>{\raggedleft\arraybackslash}p{3.2cm}|}{\makebox[3.2cm][r]{Less than Primary }}& 0.4 & 0.17 & 0.28 & 0.12 & 0.29 & 0.1 & 0.11 & 0.07 & 0.03 & 0.01 & 0.03 \\ 
  \multicolumn{1}{>{\raggedleft\arraybackslash}p{1.7cm}|}{\makebox[1.7cm][r]{Primary }}& 0.23 & 0.11 & 0.2 & 0.16 & 0.17 & 0.07 & 0.06 & 0.06 & 0.04 & 0.02 & 0.03 \\ 
  \multicolumn{1}{>{\raggedleft\arraybackslash}p{2cm}|}{\makebox[2cm][r]{Secondary }}& 0.29 & 0.51 & 0.41 & 0.49 & 0.41 & 0.52 & 0.42 & 0.64 & 0.69 & 0.59 & 0.57 \\ 
  \multicolumn{1}{>{\raggedleft\arraybackslash}p{2cm}|}{\makebox[2cm][r]{University }}& 0.07 & 0.21 & 0.12 & 0.23 & 0.14 & 0.3 & 0.41 & 0.23 & 0.25 & 0.39 & 0.38 \\ 
  Household &  &  &  &  &  &  &  &  &  &  &  \\ 
  \multicolumn{1}{>{\raggedleft\arraybackslash}p{2.7cm}|}{\makebox[2.7cm][r]{Household Size }}& 3.45 & 2.44 & 2.93 & 2.57 & 3.24 & 2.82 & 2.86 & 2.37 & 2.06 & 1.95 & 2.27 \\ 
  \multicolumn{1}{>{\raggedleft\arraybackslash}p{2.2cm}|}{\makebox[2.2cm][r]{Lives Alone }}& 0.15 & 0.3 & 0.23 & 0.27 & 0.18 & 0.2 & 0.19 & 0.28 & 0.39 & 0.29 & 0.31 \\ 
  \multicolumn{1}{>{\raggedleft\arraybackslash}p{2.9cm}|}{\makebox[2.9cm][r]{Lives with Child }}& 0.53 & 0.35 & 0.47 & 0.35 & 0.5 & 0.43 & 0.4 & 0.28 & 0.24 & 0.14 & 0.25 \\ 
  \multicolumn{1}{>{\raggedleft\arraybackslash}p{3.4cm}|}{\makebox[3.4cm][r]{Married/Cohabiting }}& 0.53 & 0.4 & 0.4 & 0.41 & 0.44 & 0.51 & 0.58 & 0.49 & 0.35 & 0.59 & 0.48 \\ 
  Age Migrated &  &  &  &  &  &  &  &  &  &  &  \\ 
  \multicolumn{1}{>{\raggedleft\arraybackslash}p{1.6cm}|}{\makebox[1.6cm][r]{24 - 49 }}& 0.77 & 0.61 & 0.69 & 0.63 & 0.79 & 0.7 & 0.7 & - & - & - & - \\ 
  \multicolumn{1}{>{\raggedleft\arraybackslash}p{2.6cm}|}{\makebox[2.6cm][r]{50 and Above }}& 0.23 & 0.39 & 0.31 & 0.37 & 0.21 & 0.3 & 0.3 & 1 & 1 & 1 & 1 \\ 
  Migration Cohort &  &  &  &  &  &  &  &  &  &  &  \\ 
  \multicolumn{1}{>{\raggedleft\arraybackslash}p{2.3cm}|}{\makebox[2.3cm][r]{Before 1965 }}& 0.03 & 0.05 & 0.03 & 0.08 & 0.02 & 0.03 & 0.04 & - & - & - & - \\ 
  \multicolumn{1}{>{\raggedleft\arraybackslash}p{2.2cm}|}{\makebox[2.2cm][r]{1965 - 1979 }}& 0.29 & 0.22 & 0.2 & 0.34 & 0.24 & 0.23 & 0.25 & - & - & - & - \\ 
  \multicolumn{1}{>{\raggedleft\arraybackslash}p{2.2cm}|}{\makebox[2.2cm][r]{1980 - 1999 }}& 0.52 & 0.43 & 0.55 & 0.38 & 0.61 & 0.47 & 0.5 & - & - & - & - \\ 
  \multicolumn{1}{>{\raggedleft\arraybackslash}p{2.1cm}|}{\makebox[2.1cm][r]{After 1999 }}& 0.2 & 0.35 & 0.27 & 0.32 & 0.19 & 0.31 & 0.25 & - & - & - & - \\ 
  Acculturation &  &  &  &  &  &  &  &  &  &  &  \\ 
  \multicolumn{1}{>{\raggedleft\arraybackslash}p{1.6cm}|}{\makebox[1.6cm][r]{Citizen }}& 0.45 & - & 0.64 & 0.75 & 0.61 & 0.67 & 0.73 & - & - & - & - \\ 
  \multicolumn{1}{>{\raggedleft\arraybackslash}p{3cm}|}{\makebox[3cm][r]{English Speakers }}& 0.64 & 0.84 & 0.62 & 0.66 & 0.77 & 0.84 & 0.9 & 0.99 & 1 & 1 & 1 \\ 
  N & 44152 & 9159 & 6541 & 13647 & 13655 & 17935 & 227703 & 120724 & 313063 & 3165675 & 94162 \\ 
   \hline
\end{tabular}
\endgroup
\end{table}


% table 2 standardized


% 2010 and 2020 mexico, pr, us

\begin{table}[ht]
\centering
\caption{Summary Statistics by Country and Sex For Hispanics in Their Native Countries} 
\begingroup\small
\begin{tabular}{l>{\raggedright\arraybackslash}p{3.2cm}|cccccc}
  \hline
Gender & Demographics & Mexico 2010 & Mexico 2020 & Puerto Rico 2010 & Puerto Rico 2020 & United States 2010 & United States 2020 \\ 
  \hline
Female & Age &  &  &  &  &  &  \\ 
   & \multicolumn{1}{>{\raggedleft\arraybackslash}p{1.5cm}|}{\makebox[1.5cm][r]{60 - 69 }} & 0.53 & 0.55 & 0.5 & 0.45 & 0.49 & 0.5 \\ 
   & \multicolumn{1}{>{\raggedleft\arraybackslash}p{1.5cm}|}{\makebox[1.5cm][r]{70 - 79 }} & 0.31 & 0.29 & 0.32 & 0.34 & 0.29 & 0.31 \\ 
   & \multicolumn{1}{>{\raggedleft\arraybackslash}p{1.5cm}|}{\makebox[1.5cm][r]{80 - 89 }} & 0.13 & 0.13 & 0.15 & 0.17 & 0.18 & 0.15 \\ 
   & \multicolumn{1}{>{\raggedleft\arraybackslash}p{1.5cm}|}{\makebox[1.5cm][r]{90 plus }} & 0.03 & 0.03 & 0.03 & 0.04 & 0.05 & 0.05 \\ 
   & Education Completed &  &  &  &  &  &  \\ 
   & \multicolumn{1}{>{\raggedleft\arraybackslash}p{3.2cm}|}{\makebox[3.2cm][r]{Less than Primary }} & 0.61 & 0.45 & 0.26 & 0.15 & 0.04 & 0.03 \\ 
   & \multicolumn{1}{>{\raggedleft\arraybackslash}p{1.7cm}|}{\makebox[1.7cm][r]{Primary }} & 0.29 & 0.38 & 0.25 & 0.21 & 0.14 & 0.09 \\ 
   & \multicolumn{1}{>{\raggedleft\arraybackslash}p{2cm}|}{\makebox[2cm][r]{Secondary }} & 0.07 & 0.11 & 0.36 & 0.44 & 0.63 & 0.62 \\ 
   & \multicolumn{1}{>{\raggedleft\arraybackslash}p{2cm}|}{\makebox[2cm][r]{University }} & 0.03 & 0.07 & 0.13 & 0.2 & 0.2 & 0.26 \\ 
   & Household &  &  &  &  &  &  \\ 
   & \multicolumn{1}{>{\raggedleft\arraybackslash}p{2.7cm}|}{\makebox[2.7cm][r]{Household Size }} & 3.56 & 3.42 & 2.23 & 2.15 & 2.01 & 2.06 \\ 
   & \multicolumn{1}{>{\raggedleft\arraybackslash}p{2.2cm}|}{\makebox[2.2cm][r]{Lives Alone }} & 0.13 & 0.14 & 0.27 & 0.3 & 0.36 & 0.33 \\ 
   & \multicolumn{1}{>{\raggedleft\arraybackslash}p{2.9cm}|}{\makebox[2.9cm][r]{Lives with Child }} & 0.58 & 0.57 & 0.29 & 0.26 & 0.17 & 0.17 \\ 
   & \multicolumn{1}{>{\raggedleft\arraybackslash}p{3.4cm}|}{\makebox[3.4cm][r]{Married/Cohabiting }} & 0.46 & 0.47 & 0.42 & 0.4 & 0.47 & 0.49 \\ 
   & N & 624788 & 975534 & 5267 & 24887 & 380399 & 2292118 \\ 
  Male & Age &  &  &  &  &  &  \\ 
   & \multicolumn{1}{>{\raggedleft\arraybackslash}p{1.5cm}|}{\makebox[1.5cm][r]{60 - 69 }} & 0.54 & 0.56 & 0.54 & 0.48 & 0.55 & 0.54 \\ 
   & \multicolumn{1}{>{\raggedleft\arraybackslash}p{1.5cm}|}{\makebox[1.5cm][r]{70 - 79 }} & 0.31 & 0.3 & 0.32 & 0.35 & 0.29 & 0.31 \\ 
   & \multicolumn{1}{>{\raggedleft\arraybackslash}p{1.5cm}|}{\makebox[1.5cm][r]{80 - 89 }} & 0.12 & 0.12 & 0.13 & 0.15 & 0.14 & 0.12 \\ 
   & \multicolumn{1}{>{\raggedleft\arraybackslash}p{1.5cm}|}{\makebox[1.5cm][r]{90 plus }} & 0.03 & 0.02 & 0.02 & 0.02 & 0.02 & 0.02 \\ 
   & Education Completed &  &  &  &  &  &  \\ 
   & \multicolumn{1}{>{\raggedleft\arraybackslash}p{3.2cm}|}{\makebox[3.2cm][r]{Less than Primary }} & 0.56 & 0.4 & 0.21 & 0.15 & 0.04 & 0.03 \\ 
   & \multicolumn{1}{>{\raggedleft\arraybackslash}p{1.7cm}|}{\makebox[1.7cm][r]{Primary }} & 0.29 & 0.37 & 0.27 & 0.24 & 0.12 & 0.08 \\ 
   & \multicolumn{1}{>{\raggedleft\arraybackslash}p{2cm}|}{\makebox[2cm][r]{Secondary }} & 0.07 & 0.1 & 0.38 & 0.43 & 0.54 & 0.56 \\ 
   & \multicolumn{1}{>{\raggedleft\arraybackslash}p{2cm}|}{\makebox[2cm][r]{University }} & 0.08 & 0.13 & 0.14 & 0.18 & 0.29 & 0.32 \\ 
   & Household &  &  &  &  &  &  \\ 
   & \multicolumn{1}{>{\raggedleft\arraybackslash}p{2.7cm}|}{\makebox[2.7cm][r]{Household Size }} & 3.7 & 3.52 & 2.4 & 2.28 & 2.16 & 2.16 \\ 
   & \multicolumn{1}{>{\raggedleft\arraybackslash}p{2.2cm}|}{\makebox[2.2cm][r]{Lives Alone }} & 0.11 & 0.11 & 0.18 & 0.22 & 0.22 & 0.23 \\ 
   & \multicolumn{1}{>{\raggedleft\arraybackslash}p{2.9cm}|}{\makebox[2.9cm][r]{Lives with Child }} & 0.56 & 0.53 & 0.24 & 0.2 & 0.14 & 0.15 \\ 
   & \multicolumn{1}{>{\raggedleft\arraybackslash}p{3.4cm}|}{\makebox[3.4cm][r]{Married/Cohabiting }} & 0.74 & 0.74 & 0.67 & 0.61 & 0.71 & 0.69 \\ 
   & N & 572708 & 887846 & 3989 & 18867 & 306158 & 1932434 \\ 
   \hline
\end{tabular}
\endgroup
\end{table}

\end{landscape}

\newpage
\section*{Acknowledgments}
\addcontentsline{toc}{section}{Acknowledgments}

The authors wish to acknowledge the statistical offices that provided
the underlying data making this research possible: National
Administrative Department of Statistics, Colombia; Office of National
Statistics, Cuba; National Statistics Office, Dominican Republic;
Department of Statistics and Censuses, El Salvador; National Institute
of Statistics, Guatemala; National Institute of Statistics, Honduras;
National Institute of Statistics, Geography, and Informatics, Mexico;
U.S. Bureau of the Census, Puerto Rico; and Bureau of the Census, United
States.

\newpage

\newpage

\section*{References}\label{references}
\addcontentsline{toc}{section}{References}

\phantomsection\label{refs}
\begin{CSLReferences}{1}{0}
\bibitem[\citeproctext]{ref-aldridge_global_2018}
Aldridge, Robert W., Laura B. Nellums, Sean Bartlett, Anna Louise Barr,
Parth Patel, Rachel Burns, Sally Hargreaves, et al. 2018. {``Global
Patterns of Mortality in International Migrants: A Systematic Review and
Meta-Analysis.''} \emph{Lancet (London, England)} 392 (10164): 2553--66.
\url{https://doi.org/10.1016/S0140-6736(18)32781-8}.

\bibitem[\citeproctext]{ref-angel_aging_2017}
Angel, Jacqueline L., William Vega, and Mariana López-Ortega. 2017.
{``Aging in {Mexico}: {Population} {Trends} and {Emerging} {Issues}.''}
\emph{The Gerontologist} 57 (2): 153--62.
\url{https://doi.org/10.1093/geront/gnw136}.

\bibitem[\citeproctext]{ref-bahar_often_2024}
Bahar, Dany. 2024. {``The {Often} {Overlooked} {`{Pull}'} {Factor}:
{Border} {Crossings} and {Labor} {Market} {Tightness} in the {US},''}
June.
\url{https://www.cgdev.org/publication/often-overlooked-pull-factor-border-crossings-and-labor-market-tightness-us}.

\bibitem[\citeproctext]{ref-batalova_central_2021}
Batalova, Jeanne Batalova Erin Babich and Jeanne. 2021. {``Central
{American} {Immigrants} in the {United} {States}.''}
\emph{Migrationpolicy.org}.
\url{https://www.migrationpolicy.org/article/central-american-immigrants-united-states-2019}.

\bibitem[\citeproctext]{ref-capielo_rosario_conceptualizing_2023}
Capielo Rosario, Cristalís, Tristan Mattwig, Kyana D. Hamilton, and
Brenton Wejrowski. 2023. {``Conceptualizing {Puerto} {Rican} Migration
to the {United} {States}.''} \emph{Current Opinion in Psychology} 51
(June): 101584. \url{https://doi.org/10.1016/j.copsyc.2023.101584}.

\bibitem[\citeproctext]{ref-causa_should_2020}
Causa, O., and J. Pichelmann. 2020. {``Should {I} Stay or Should {I} Go?
{Housing} and Residential Mobility Across {OECD} Countries.''}
\emph{OECD} 1626 (October).
https://doi.org/\url{https://doi.org/10.1787/d91329c2-en}.

\bibitem[\citeproctext]{ref-cervantes_salvador_2018}
Cervantes, Cecilia Menjívar and Andrea Gómez. 2018. {``El {Salvador}:
{Civil} {War}, {Natural} {Disasters}, and {Gang} {Violence} {Drive}
{Migration}.''} \emph{Migrationpolicy.org}.
\url{https://www.migrationpolicy.org/article/el-salvador-civil-war-natural-disasters-and-gang-violence-drive-migration}.

\bibitem[\citeproctext]{ref-clemens_place_2009}
Clemens, Michael A., Claudio E. Montenegro, and Lant Pritchett. 2009.
{``The {Place} {Premium}: {Wage} {Differences} for {Identical} {Workers}
{Across} the {U}.{S}. {Border}.''} \{SSRN\} \{Scholarly\} \{Paper\}.
Rochester, NY. \url{https://doi.org/10.2139/ssrn.1211427}.

\bibitem[\citeproctext]{ref-duany_send_2010}
Duany, Jorge. 2010. {``To {Send} or {Not} to {Send}: {Migrant}
{Remittances} in {Puerto} {Rico}, the {Dominican} {Republic}, and
{Mexico}.''} \emph{The ANNALS of the American Academy of Political and
Social Science} 630 (1): 205--23.
\url{https://doi.org/10.1177/0002716210368111}.

\bibitem[\citeproctext]{ref-duany_cuban_2017}
---------. 2017. {``Cuban {Migration}: {A} {Postrevolution} {Exodus}
{Ebbs} and {Flows}.''} \emph{Migrationpolicy.org}.
\url{https://www.migrationpolicy.org/article/cuban-migration-postrevolution-exodus-ebbs-and-flows}.

\bibitem[\citeproctext]{ref-fitzgerald_sociology_2018}
FitzGerald, David Scott, and Rawan Arar. 2018. {``The {Sociology} of
{Refugee} {Migration}.''} \emph{Annual Review of Sociology} 44 (Volume
44, 2018): 387--406.
\url{https://doi.org/10.1146/annurev-soc-073117-041204}.

\bibitem[\citeproctext]{ref-gonzalez_current_2024}
González, Juan. 2024. {``The {Current} {Migrant} {Crisis}: {How}
{U}.{S}. {Policy} {Toward} {Latin} {America} {Has} {Fueled} {Historic}
{Numbers} of {Asylum} {Seekers} - {New} {Labor} {Forum}.''}
\url{https://newlaborforum.cuny.edu/2024/05/01/the-current-migrant-crisis-how-u-s-policy-toward-latin-america-has-fueled-historic-numbers-of-asylum-seekers/}.

\bibitem[\citeproctext]{ref-hanson_us_2023}
Hanson, Gordon, Pia Orrenius, and Madeline Zavodny. 2023. {``{US}
{Immigration} from {Latin} {America} in {Historical} {Perspective}.''}
\emph{Journal of Economic Perspectives} 37 (1): 199--222.
\url{https://doi.org/10.1257/jep.37.1.199}.

\bibitem[\citeproctext]{ref-henao_subjective_2023}
Henao, Carolina, Jenny Paola Lis-Gutiérrez, and Manuel Ignacio
Balaguera. 2023. {``Subjective {Quality} of {Life} in {Latin}
{American}.''} \emph{Salud, Ciencia y Tecnología - Serie de
Conferencias} 2 (September): 384--84.
\url{https://doi.org/10.56294/sctconf2023384}.

\bibitem[\citeproctext]{ref-jonas_centaurs_2018}
Jonas, Susanne. 2018. \emph{Of {Centaurs} {And} {Doves}: {Guatemala}'s
{Peace} {Process}}. New York: Routledge.
\url{https://doi.org/10.4324/9780429498596}.

\bibitem[\citeproctext]{ref-kritz_international_1992}
Kritz, Mary M., Lin Lean Lim, and Hania Zlotnik. 1992.
\emph{International {Migration} {Systems}\,: {A} {Global} {Approach}}.
Seminar on {International} {Migration} {Systems}. Oxford: Clarendon
Press.

\bibitem[\citeproctext]{ref-larotta_silva_determinantes_2019}
Larotta Silva, Sonia Patricia. 2019. {``Determinantes Para La Migración
Internacional de Colombianos Entre 1990-2015 a Partir de Un Modelo
Gravitacional.''} \emph{Territorios}, no. 41 (December): 69--100.
\url{https://doi.org/10.12804/revistas.urosario.edu.co/territorios/a.7414}.

\bibitem[\citeproctext]{ref-llibre-guerra_caribbean-american_2021}
Llibre-Guerra, Jorge J, Jing Li, Amal Harrati, Ivonne Jiménez-Velazquez,
Daisy M Acosta, Juan J Llibre-Rodriguez, Mao-Mei Liu, and William H Dow.
2021. {``The {Caribbean}-{American} {Dementia} and {Aging} {Study}
({CADAS}): {A} Multinational Initiative to Address Dementia in
{Caribbean} Populations.''} \emph{Alzheimer's \& Dementia} 17 (S7):
e053789. \url{https://doi.org/10.1002/alz.053789}.

\bibitem[\citeproctext]{ref-massey_evaluation_1994}
Massey, Douglas S., Joaquín Arango, Graeme Hugo, Ali Kouaouci, Adela
Pellegrino, and J. Edward Taylor. 1994. {``An {Evaluation} of
{International} {Migration} {Theory}: {The} {North} {American}
{Case}.''} \emph{Population and Development Review} 20 (4): 699--751.
\url{https://doi.org/10.2307/2137660}.

\bibitem[\citeproctext]{ref-mcauliffe_who_2024}
McAuliffe, Marie, Celine Bauloz, and Adrian Kitimbo. 2024. {``Who
Migrates Internationally and Where Do They Go? {International}.''}
International Organization for Migration.
\url{https://worldmigrationreport.iom.int/what-we-do/world-migration-report-2024-chapter-4/who-migrates-internationally-and-where-do-they-go-international-migration-globally-between-1995-2020}.

\bibitem[\citeproctext]{ref-noauthor_nationwide_2024}
{``Nationwide {Encounters} {\textbar} {U}.{S}. {Customs} and {Border}
{Protection}.''} 2024. \emph{U.S. Customs and Border Protection}.
\url{https://www.cbp.gov/newsroom/stats/nationwide-encounters}.

\bibitem[\citeproctext]{ref-nevado-pena_improving_2019}
Nevado-Peña, Domingo, Víctor-Raúl López-Ruiz, and José-Luis
Alfaro-Navarro. 2019. {``Improving Quality of Life Perception with {ICT}
Use and Technological Capacity in {Europe}.''} \emph{Technological
Forecasting and Social Change} 148 (November): 119734.
\url{https://doi.org/10.1016/j.techfore.2019.119734}.

\bibitem[\citeproctext]{ref-noe-bustamante_key_2023}
Noe-Bustamante, Mohamad Moslimani and Luis, Jeffrey S. Passel. 2023.
{``Key Facts about {U}.{S}. {Latinos} for {National} {Hispanic}
{Heritage} {Month}.''} \emph{Pew Research Center}.
\url{https://www.pewresearch.org/short-reads/2023/09/22/key-facts-about-us-latinos-for-national-hispanic-heritage-month/}.

\bibitem[\citeproctext]{ref-passel_what_2024}
Passel, Mohamad Moslimani and Jeffrey S. 2024. {``What the Data Says
about Immigrants in the {U}.{S}.''} \emph{Pew Research Center}.
\url{https://www.pewresearch.org/short-reads/2024/09/27/key-findings-about-us-immigrants/}.

\bibitem[\citeproctext]{ref-pine_working_2008}
Pine, Adrienne. 2008. \emph{Working {Hard}, {Drinking} {Hard}: {On}
{Violence} and {Survival} in {Honduras}}. 1st ed. University of
California Press.
\url{https://www.jstor.org/stable/10.1525/j.ctt1pppbx}.

\bibitem[\citeproctext]{ref-ruggles_ipums_2024}
Ruggles, Streven, Lara Cleveland, Sula Sarkar, and Matthew Sobek. 2024.
{``{IPUMS} {International}.''}
https://doi.org/\url{https://doi.org/10.18128/D020.V7.5}.

\bibitem[\citeproctext]{ref-silva_violence_2014}
Silva, Adriana C., and Douglas S. Massey. 2014. {``Violence, {Networks},
and {International} {Migration} from {Colombia} - {Silva} - 2015 -
{International} {Migration} - {Wiley} {Online} {Library}''} 53 (5):
162--78. https://doi.org/\url{https://doi.org/10.1111/imig.12169}.

\bibitem[\citeproctext]{ref-tanco_dramatic_2023}
Tanco, Ariel G. Ruiz Soto, Valerie Lacarte. 2023. {``In a {Dramatic}
{Shift}, the {Americas} {Have} {Become} a {Leading} {Migration}
{Destination}.''} \emph{Migrationpolicy.org}.
\url{https://www.migrationpolicy.org/article/latin-america-caribbean-immigration-shift}.

\bibitem[\citeproctext]{ref-valentine_migration_2017}
Valentine, Jessa Lewis, Brad Barham, Seth Gitter, and Jenna Nobles.
2017. {``Migration and the {Pursuit} of {Education} in {Southern}
{Mexico}.''} \emph{Comparative Education Review} 61 (1): 141--75.
\url{https://doi.org/10.1086/689615}.

\bibitem[\citeproctext]{ref-winkler_how_2016}
Winkler, Hernan. 2016. {``How Does the Internet Affect Migration
Decisions?: {Applied} {Economics} {Letters}: {Vol} 24 , {No} 16 - {Get}
{Access}.''} \emph{Applied Economics Letters} 24 (16): 1194--98.
https://doi.org/\url{https://doi.org/10.1080/13504851.2016.1265069}.

\end{CSLReferences}




\end{document}
